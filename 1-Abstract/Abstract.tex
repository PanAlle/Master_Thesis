\documentclass[../main.tex]{subfiles}
\begin{document}
Software engineering in the field of automotive is a continuously developing field. The lines of code embedded in the ECU control every aspect of the vehicle, from heating in the cockpit to the output speed, all the way to ADAS capabilities.\\
The rising in complexity required a fast growth in the fields of software development for the Automotive. The expression of this complexity can be grasped by the description of the processes introduced in the thesis. The first part of the work discuss and explains how the process of developing software is done in the automotive world, following the workflow used at BMW.\\
The focus then switches to the main topic of the work, the development of a data-pipeline for the analysis of data relative to software versions for the ECU. During the development and integration of software in the ECU a great amount of data is generated. Most of the time the data is used by single departments working on a specific task or, in some cases just by the single developer. A centralized overview on the overall data in the process is lacking.\\
Starting from this request the thesis reports the process of developing an infrastructure to actively store data in a database and create a dashboard to visualize it. The main goal of the project is to create an easy accessible dashboard with continuously updating data for a real time feedback on the software development process.\\
The workflow starts with the creation of a MVP (minimum valuable product) and continue with the set up of the infrastructure that supports the flow of data from the software versions to the database and at the end to the dashboard.\\

\cleardoublepage
\end{document}