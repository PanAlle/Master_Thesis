\documentclass[../main.tex]{subfiles}
\begin{document}

% E' consentito agli studenti che ne facciano richiesta al proprio relatore, di formulare la tesi in lingua inglese e in ogni altra lingua straniera di uno stato dell’Unione Europea, purché il testo venga preceduto da un'ampia sintesi (dell'ordine di 10 pagine) dei contenuti in lingua italiana. All'atto della presentazione dei prescritti moduli in Segreteria studenti, il titolo della tesi dovrà essere definito sia in lingua italiana sia nella lingua straniera ed entrambe le formulazioni dovranno essere riportate sull'intestazione della tesi stessa.
\section{Introduzione}
Nel seguente elaborato verrà trattato lo sviluppo di un pipeline dati per l'analisi del software della centralina controllo motore. Il progetto è stato svolto presso l´azienda BMW di Monaco di Baviera.\\
L´elaborato parte con lo spiegare la struttura utilizzata per il processo di sviluppo del software. Il primo capitolo infatti tratta delle metodologie di organizzazione del lavoro che concretizzano la filosofia Agile. In particolare il focus ricade sulla metodologia “Scrum”, la quale, seppur di recente sviluppo, è ormai un cardine all'interno dei reparti di sviluppo software. 
Dopo i concetti chiave del “modo di lavorare”, utilizzato nel progetto, l'elaborato continua presentano la componente teorica che si cela dietro lo sviluppo software. In questa sezione vengono trattate le basi legate al funzionamento di compilatori, per poi passare a una parte piú legata al funzionamento delle centraline auto viste da un punto di vista Software.
Continuando viene poi affrontato il problema sviluppato nella tesi. Ovvero la creazione di una pipeline di dati per l'analisi di software centraline.\\
Lo sviluppo software, nel campo del controllo motore si puó definire contraddistinto da due poli. Il primo é lo sviluppo software per la centralina stessa. Questa contiene infatti non solamente codice, ma numerose informazioni di vario genere. Sono infatti presenti informazioni sull´architettura stessa della centralina, su come il software si interfaccia con i componenti della centralina stessa e dell veicolo in generale. Più in dettaglio sono presenti informazioni sullo scheduling, interfaccia AUTOSASR,  gestione CAN, informazioni sui protocolli di comunicazione. All´altro polo della progettazione c´é lo sviluppo di tutte funzioni che  gestiscono il motore stesso, dai sensori della temperatura, all´indicatore dell´olio fino al più complessa gestione dell´iniezione. Tutte queste funzioni sono sviluppate secondo la logica "Model Based Design" in Matlab. Ovviamente il blocco Simulink, prodotto di questa parte del processo, non può essere direttamente caricato sulla centralina. A collegare questi poli si trova quello che puó essere esemplificato sotto la voce compilatore. Il compito del compilatore é quello di ricevere come input il blocco funzione e generare come output codice leggibile dalla centralina.\\ Il compilatore é sviluppato in BMW in casa, basato in Python, e si occupa di generare non solo il codice per la centralina ma anche tutto ciò che c'è intorno, a partire da informazioni sul software generato per il costruttore della centralina (AUTOSAR) a informazioni più dirette per gli sviluppatori come la qualità con cui le funzioni motore interagiscono tra di loro. Questo software data la complessità genera una grande quantità di dati che presi singolarmente sono utili solamente a un determinato dipartimento. Il progetto si sviluppa nell´inserirsi all´interno di questo software per raccogliere i vari dati. La raccolta viene fatta in un database, il quale deve essere gestito direttamente dal software stesso. A partire dai dati presenti sul database lo step finale è quella di creare una visualizzazione dei dati tramite programmi di BI (visualizzazioni gratifiche). Il fatto di poter avere una visualizzazione d'insieme dei dati giova agli sviluppatori in quanto velocizza molto il processo di confronto dati provenienti da ambiti diversi. Inoltre, aspetto non certamente secondario permette una chiara visualizzazione del dato, il quale la maggior parte delle volte é fornito in output dal compilatore in formati "\textit{machine frinedly}" (.xml, .json) i quali non sono immediati alla lettura.\\Il termine pipeline deriva proprio da questo, il software, venendo eseguito automaticamente aggiorna direttamente la Dashboard di visualizzazione dati, creando un vero e proprio “\textit{tubo}” dove i dati scorrono(vengono aggiornati e caricati in un database) in modo automatico. 

\cleardoublepage
\end{document}