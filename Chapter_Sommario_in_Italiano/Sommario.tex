\documentclass[../main.tex]{subfiles}
\begin{document}

% E' consentito agli studenti che ne facciano richiesta al proprio relatore, di formulare la tesi in lingua inglese e in ogni altra lingua straniera di uno stato dell’Unione Europea, purché il testo venga preceduto da un'ampia sintesi (dell'ordine di 10 pagine) dei contenuti in lingua italiana. All'atto della presentazione dei prescritti moduli in Segreteria studenti, il titolo della tesi dovrà essere definito sia in lingua italiana sia nella lingua straniera ed entrambe le formulazioni dovranno essere riportate sull'intestazione della tesi stessa.
\section{Introduzione}
Nel seguente elaborato verrà trattato lo sviluppo di un pipeline dati per l'analisi di dati legati alla sezione centralina controllo motore. 
Il progetto è stato svolto presso l´azienda BMW di Monaco di Baviera. 
La trattazione parte con spiegare il modello utilizzato nello sviluppo software. Il primo capitolo infatti tratta delle metodologie di organizzazione del lavoro che concretizzano la filosofia Agile. In particolare il focus ricade sulla metodologia “Scrum”. La metodologia, seppur di recente sviluppo, è ormai un cardine all'interno dei reparti di sviluppo software. 
Dopo i concetti chiave del “modo di lavorare”, utilizzato nel progetto, l'elaborato continua presentano la componente teorica che si cela dietro lo sviluppo software. In questa sezione vengono trattate le basi legate al funzionamento di compilatori, per poi passare a una parte piú legata al funzionamento delle centraline auto. 
Continuando viene poi affrontato il problema sviluppato nella tesi. Ovvero la creazione di una pipeline di dati per l'analisi di software centraline. 
Il punto di partenza per lo sviluppo al fine di analizzare il processo é la centralina stessa. Questa contiene infatti non solamente codice, ci sono infatti informazioni sullárchitettura stessa della centralina, il codice non è semplice codice ma si basa su standard AUTOSAR. Inoltre sono presenti informazioni sullo scheduling, gestione CAN, informazioni sui protocolli di comunicazione. Dall´altra parte del progetto c'è lo sviluppo di funzioni per gestire il motore stesso. Queste sono sviluppate a partire da blocchi di Matlab. Queste funzioni definiscono come avviene il controllo di tutto quello che passa in centralina. Ovviamente il blocco Simulink non può essere direttamente caricato sulla centralina. Nel mezzo di questo sistema si trova quello che puó essere esemplificato sotto la parola compilatore. Questo ricevendo come input il blocco funzione genera come output codice leggibile dalla centralina. Questo software in BMW è scritto in Python, e si occupa di generare non solo il codice per la centralina ma anche tutto ció che c'è intorno, a partire da informazioni sul software generato per il costruttore della centralina a informazioni più dirette per gli sviluppatori come la qualità con cui le funzioni motore interagiscono tra di loro. Questo software data la complessità genera una grande quanti dati che presi singolarmente sono utili solamente a un determinato dipartimento. La tesi si sviluppa nell´inserirsi all´interno di questo software per raccogliere i vari dati. La raccolta viene fatta in un database, il quale deve essere gestito direttamente dal software stesso. A partire dai dati presenti sul database lo step finale è quella di creare una visualizzazione dei dati tramite programmi di  BI. Il fatto di poter avere una visualizzazione d'insieme dei dati giova agli sviluppatori perché risulta molto più veloci visualizzare dati provenienti da ambiti diversi e confrontarli in modo facile e diretto. Il termine pipeline v deriva proprio da questo, che il software, venendo eseguito automaticamente aggiorna direttamente la Dashboard di visualizzazione dati, creando una vera propria “tubo” dove i dati scorrono(vengono aggiornati e caricati in un database) in modo automatico. 

\cleardoublepage
\end{document}