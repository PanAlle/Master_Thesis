\documentclass[../main.tex]{subfiles}
\begin{document}

% E' consentito agli studenti che ne facciano richiesta al proprio relatore, di formulare la tesi in lingua inglese e in ogni altra lingua straniera di uno stato dell’Unione Europea, purché il testo venga preceduto da un'ampia sintesi (dell'ordine di 10 pagine) dei contenuti in lingua italiana. All'atto della presentazione dei prescritti moduli in Segreteria studenti, il titolo della tesi dovrà essere definito sia in lingua italiana sia nella lingua straniera ed entrambe le formulazioni dovranno essere riportate sull'intestazione della tesi stessa.
\section{Abstract - Sintesi italiana}
Lo sviluppo software in ambito Automotive é un campo in continua espansione. Le linee di codice all'interno della centralina controllano tutti gli aspetti del veicolo, dalla temperatura nell'abitacolo, alla velocità su strada fino ai sistemi di guida autonoma ADAS. 
Le linee di codice presenti nei veicoli sono passate da quasi zero a qualche milione in un arco temporale di circa 50 anni. La crescita nella complessità ha richiesto una grande espansione nel campo dello sviluppo software per applicazioni autoveicolo. La complessità dietro i processi traspare nella prima parte della tesi, dove vengono analizzati i processi di sviluppo software all'interno del mondo Automotive, seguendo le metodologie usate presso \gls{BMW}.\\
Il focus si sposta poi sull'argomento principale della tesi, ovvero lo sviluppo di una pipeline dati per l'analisi di informazioni relative alle versioni di software per la centralina. Durante il ciclo di sviluppo software nella centralina una grande quantità di dati viene generata. Molto spesso questi dati vengono usati solamente dai dipartimenti responsabili dello sviluppo di una certa funzionalità e a volte, solamente dal singolo sviluppatore. Manca un sistema centralizzato che fornisca informazioni su tutto il processo.\\
Partendo da questa richiesta la tesi riporta il processo di sviluppo di un infrastruttura che possa attivamente salvare informazioni in un database e fornire un feedback agli sviluppatori tramite una visualizzazione grafica dei dati (Dashboard). Lo scopo del progetto é proprio questo, ovvero creare una Dashboard che possa fornire in tempo reale informazioni sullo stato di avanzamento dello sviluppo del software.\\
Il lavoro riportato nella tesi, parte con la creazione di un MVP, ovvero un primo prototipo atto a verificare la fattibilità della proposta e raccogliere un primo feedback sulla Dashboard. Dopodiché lo sviluppo continua con la creazione di una infrastruttura articolata che possa supportare il flusso id dati dal software, al database fino alla Dashboard. 

\section{Introduzione - Sintesi italiana}
Nel capitolo di introduzione viene trattato il tema del sistema automobile. Partendo da una semplificazione del modello automobile, riportato come semplice sistema ad anello chiuso la trattazione si sviluppa fino a definire la più complessa dicotomia elettronica meccanica nel veicolo, come riportato in Figura \ref{fig:schemaablocchi}.
\begin{figure}[ht]
        \begin{center}
  \begin{tikzpicture}[auto, node distance=3cm,>=latex', scale=0.7,transform shape]
            \node [input1, name=input] {};
            \node [sum1, right=of input] (sum) {};
            \node [sum, right=of sum] (sum_A) {};
            \node [block, right=of sum_A] (controller) {$Electronics$};
            \node [block, right=of controller] (mechanics) {$Mechanics$};
            \node [output1, right=of mechanics] (output) {};
            \draw [draw,->] (input) -- node {$User\; inputs$} (sum);
            \draw [->] (sum) -- node {} (sum_A);
            \draw [->] (sum_A) -- node {} (controller);
            \draw [->] (controller) -- node {} (mechanics);
            \draw [->] (mechanics) -- node [name=y] {$(speed,\; position..)$}(output);
            \draw [->] (mechanics) -- ++ (0,-2) -| node [pos=0.99] {$-$} (sum_A);
            \draw [->] (y) -- ++ (0,-4) -| node [pos=0.99] {$-$} (sum);
            % \node [container,fit=(controller) (mechanics) (sum_A)] (container) {};
            \end{tikzpicture}
        \end{center}
        \caption{Schema a blocchi di un'aumtomobile}
        \label{fig:schemaablocchi}
    \end{figure}
La trattazione continua trattando il tema del software per centraline. In generale il software per \gls{ECU} può essere paragonato a software per sistemi embedded di tipo Real Time, ovvero dove la correttezza della risposta dipende non solo da una correttezza sul piano logico ma anche sul piano temporale. Con questo vengono di seguito introdotte le problematiche tipiche di sistemi real time, in modo da creare un'idea della complessità delle informazioni presenti nel software. 
Nella parte finale del capitolo viene poi introdotto il concetto di compilatore, il quale ha la funzione di ricevere come input il codice, sviluppato in Matlab che definisce il funzionamento della centralina e generare come output il codice compilato che potrà poi essere letto dall'organo centralina stesso.//
Il problema trattato nella tesi si lega a questo argomento. Infatti durante il processo di compilazione codice, il quale in \gls{BMW} è fatto da un compilatore sviluppato internamente, porta alla generazione di una grande mole di data riguardanti lo stato del processo. In generale differenti dipartimenti e differenti persone si occupano di singoli parti di questo processo. I dati non sono quindi solamente rilevanti in termine di mole ma sono anche divisi su differenti persone, rendendo complesso avere una visione d'insieme sullo stato del processo. A questo proposito l'obiettivo della tesi è quello di sviluppare una struttura che possa salvare i dati, processarli e fornirli a tutti gli utenti tramite una visualizzazione grafica. L'obiettivo di questo progetto è quello di migliorare il feedback per i singoli sviluppatori e fornire una visione d'insieme mancante. 
\section{}
\cleardoublepage
\end{document}