\documentclass[../main.tex]{subfiles}
\begin{document}
%Release v1.00
%test  1.00
The following chapter introduces the Agile philosophy, including a comparison with traditional waterfall methods. The chapter then continues with the definition of Scrum methodology, giving a overview of the tools, processes and roles related to it. 
\section{The Agile philosophy}
By reading the previous lines we need to put extra effort into analyzing each word. 
\textit{Agile} has been coupled from the beginning of the Chapter only with \textit{philosophy}.  
The word philosophy emphasises the fact that we are referring to a way of thinking but also on the other hand that we are not trying to give any method to support this way of thinking. Then we have the word Agile, which firmly add dynamism to this way of thinking. 
The Agile philosophy is this, a dynamic way to approach project management. 
\subsection{History of the Agile methodology}
Agile as \cite{inproceedings}


\section{Scrum}
\subsection{Scrum Process}
\subsection{Scrum tools}
\subsection{Scrum roles}
\section{Direct application of Agile for software development}
\cleardoublepage
\end{document}