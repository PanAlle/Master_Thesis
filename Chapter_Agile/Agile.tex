\documentclass[../main.tex]{subfiles}
\begin{document}
%Release v1.00
%test
The following chapter introduces the Agile philosophy, including its principles. The chapter then continues with the definition of Scrum methodology, giving a overview of the tools, processes and roles related to it. 
By tracing back in the past the main goal of each company, except from the non-profit companies has always been the one of creating economical value by the development of products. The main approach in the creation of revenues in the traditional methodologies was the one of reducing costs.\\
Unlikely the previous attempts modern methodologies, such as Lean, Agile and Design Thinking aim to a maximization of revenues by a maximization of value inputted into the product.\\
\section{Modern approaches to product development}
The stories of modern approaches to the product development starts in the early 90´in the Us. the U.S. Automotive industry was suffering from a fierce competition from deviated form the Japanese car industry. The Japanese cars required not only half the time to develop, but were subject to half the defect that a U.S manufactured car had.\\
Although at first this great differences was seen as a cultural difference, a comprehensive survey of automobile firms by Womack, Jones, and Roos (1990) changed that belief. The survey brought to light the a new method for design and manufacturing, later named \textit{lean} that emphasizes flexibility and customer value, rather than batch and queue process of mass production. 
\section{The concept of Agile}
\subsection{History of the Agile methodology}
\section{Scrum}
\subsection{Scrum Process}
\subsection{Scrum tools}
\subsection{Scrum roles}
\section{Direct application of Agile for software development}
\cleardoublepage
\end{document}