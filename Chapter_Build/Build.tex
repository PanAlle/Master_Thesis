\documentclass[../main.tex]{subfiles}
\begin{document}

\tikzstyle{block} = [draw, rectangle, text width=2cm, text centered, minimum height=1.2cm, node distance=5cm]
\tikzstyle{bigblock} = [draw, rectangle, text width=2cm, text centered, minimum height=2.2cm, minimum width=3.2cm, node distance=5cm]

\section{Introduction}
In the following chapter the theory behind compiler and the build process is introduced. At the end a introduction over the Sconc build framework is reported. The Scons framework is the basic structure upon which the generation of Ecu code is based. 

\section{Compiler}
A compiler is a complex machine that bridges the gap between human readable code and computer readable code.\\
In a model based manner the compiler takes as input source code in a high-level programming language (such as C, C++, Python) and return as a output code in a low level language that compose the executable program. 
\begin{figure}[H]
  \centering
\begin{tikzpicture}
    \node [block, name=text1] {Source code};
    \node [bigblock, right of=text1] (text2) {Compiler};
    \node [block, right of=text2] (text3) {Executable code};
    \draw [->] (text1) -- (text2);
    \draw [->] (text2) -- node {} (text3);
\end{tikzpicture}
\end{figure}
In a executable program generated by the compiler we find a list of instruction to follow all written in binary (machine code). The instruction in binary code are needed to connect together the correct circuits in the CPU, by mean of activating (1) ore deactivating (0) certain transistors, and therefor completing the instructions. Since the CPU can only perform operation such as memory read-wright and basic math the compiler does not only need to translate the source code in binary code, but has also to execute a series of operation to adapt the complexity to the CPU. 
\cleardoublepage
\end{document}