\documentclass[../main.tex]{subfiles}
\begin{document}
Lo sviluppo software in ambito Automotive é un campo in continua espansione. Le linee di codice all'interno della centralina controllano tutti gli aspetti del veicolo, dalla temperatura nell'abitacolo, alla velocità su strada fino ai sistemi di guida autonoma ADAS. 
Le linee di codice presenti nei veicoli sono passate da quasi zero a qualche milione in un arco temporale di circa 50 anni. La crescita nella complessità ha richiesto una grande espansione nel campo dello sviluppo software per applicazioni autoveicolo. La complessità dietro i processi traspare nella prima parte della tesi, dove vengono analizzati i processi di sviluppo software all'interno del mondo Automotive, seguendo le metodologie usate presso BMW.\\
Il focus si sposta poi sull'argomento principale della tesi, ovvero lo sviluppo di una pipeline dati per l'analisi di informazioni relative alle versioni di software per la centralina. Durante il ciclo di sviluppo software nella centralina una grande quantità di dati viene generata. Molto spesso questi dati vengono usati solamente dai dipartimenti responsabili dello sviluppo di una certa funzionalità e a volte, solamente dal singolo sviluppatore. Manca un sistema centralizzato che fornisca informazioni su tutto il processo.\\
Partendo da questa richiesta la tesi riporta il processo di sviluppo di un infrastruttura che possa attivamente salvare informazioni in un database e fornire un feedback agli sviluppatori tramite una visualizzazione grafica dei dati (Dashboard). Lo scopo del progetto é proprio questo, ovvero creare una Dashboard che possa fornire in tempo reale informazioni sullo stato di avanzamento dello sviluppo del software.\\
Il lavoro riportato nella tesi, parte con la creazione di un MVP, ovvero un primo prototipo atto a verificare la fattibilità della proposta e raccogliere un primo feedback sulla Dashboard. Dopodiché lo sviluppo continua con la creazione di una infrastruttura articolata che possa supportare il flusso id dati dal software, al database fino alla Dashboard. 

\cleardoublepage
\end{document}